\documentclass[11pt]{letter}

\usepackage[hmargin={1.0in,1.0in},%
            vmargin={1.0in,1.0in},%
            nohead,%
            nofoot,%
            ]{geometry}                                 % the page layout without fancyhdr
\pagestyle{empty}
\usepackage[shortlabels]{enumitem}

\begin{document}
\address{Ross Parker \\
Department of Mathematics \\
Southern Methodist University \\
Dallas, TX 75275 \\
\texttt{rhparker@smu.edu}}%
\signature{Ross Parker}
\begin{letter}{Editor, Physica D}

\opening{Dear Editor,}

On behalf of my co-author Alejandro Aceves, I would like to submit our second revision of the article ``Multi-pulse solitary waves in a fourth-order nonlinear {S}chr{\"o}dinger equation'' for consideration of publication in Physica D. All of the suggestions for improvement from the reviewer have been systematically taken into careful consideration and incorporated into the revision, as noted below. The portions of the manuscript which have been revised are indicated using red text. Given the improvements made in accordance with the requests of the referees, we hope that you will now find the manuscript to be suitable for publication in Physica D. We will be sincerely looking forward to your editorial decision.

\begin{enumerate}[(a)]
\item \emph{On page 2 there is the phrase, ``finite set of nearby eigenvalues''. If it is known, the authors should state the size of this set. If nothing else, is there an upper bound on the size?}

\item \emph{On page 2 there is the phrase, ``reduce the eigenvalue problem for an $n$-pulse to an $n\times n$ matrix equation.'' This is unclear. Perhaps something like, ``reduce the problem of finding the eigenvalues near the origin associated with an $n$-pulse to finding the eigenvalues of an $n\times n$ matrix'', would be better.} This has been edited for clarity. Lin's method reduces the multi-pulse eigenvalue problem to finding nontrivial solutions to a $n \times n$ matrix equation. 

\item \emph{On page 2 the following sentence should perhaps start, ``Solving this matrix eigen- value problem...''}

\item \emph{Just above Remark 1 on page 4 there is the phrase, ``one of these solutions''. What is meant by ``one'' here? As far as I can tell, there is an entire family of solutions in $(\beta_2, \beta_4, \omega)$-space.} Although there an entire family of these solutions, reference [29] considers a solution for a specific set of $(\beta_2, \beta_4, \omega)$, which is now listed.

\item \emph{On the bottom of page 5 I think there is a typo in the red-lettering. I think $w$ needs to be replaced by $\omega$.} This has been corrected in both places on the bottom of page 5.

\item \emph{Just below equation (20) I think $L_0$ needs to be replaced by $L_0^2$.} This is correct and has been fixed.

\item \emph{Regarding Figure 10:}

\begin{itemize}
\item \emph{what is the size of the unstable eigenvalue associate with the two pulse?}

\item \emph{in the top left panel are the oscillations due to the small purely imaginary eigenvalues associated with the double pulse, or are they associated with the internal mode of the base pulse?}

\item \emph{in the top right panel it appears as if the individual pulses hold their shape as they continue to separate. So, in some sense what is being seen is a 2-soliton instead of a 2-pulse, and each soliton is traveling at a constant speed. If you expanded the spatial interval to $[-20, 20]$, and also extended the time interval, would it be the case that each pulse continues to travel at constant speed?}

\item \emph{in the second row why plot $Re(u)$ instead of $|u|$? This seems to be an unusual choice, and I am not sure what is gained by making it.}

\end{itemize}

\end{enumerate}

\closing{Sincerely,}

\end{letter}
\end{document}
