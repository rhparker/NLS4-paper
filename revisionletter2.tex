\documentclass[11pt]{letter}

\usepackage[hmargin={1.0in,1.0in},%
            vmargin={1.0in,1.0in},%
            nohead,%
            nofoot,%
            ]{geometry}                                 % the page layout without fancyhdr
\pagestyle{empty}
\usepackage[shortlabels]{enumitem}

\begin{document}
\address{Ross Parker \\
Department of Mathematics \\
Southern Methodist University \\
Dallas, TX 75275 \\
\texttt{rhparker@smu.edu}}%
\signature{Ross Parker}
\begin{letter}{Editor, Physica D}

\opening{Dear Editor,}

On behalf of my co-author Alejandro Aceves, I would like to submit our second revision of the article ``Multi-pulse solitary waves in a fourth-order nonlinear {S}chr{\"o}dinger equation'' for consideration of publication in Physica D. All of the suggestions for improvement from the reviewer have been systematically taken into careful consideration and incorporated into the revision, as noted below. The portions of the manuscript which have been revised are indicated using red text. Given the improvements made in accordance with the requests of the referees, we hope that you will now find the manuscript to be suitable for publication in Physica D. We will be sincerely looking forward to your editorial decision.

\begin{enumerate}
\item \emph{There is a need for a definite statement about the number of zero eigenvalues of these multi-pulse solutions in Corollary 1 and Corollary 2 which needs to be supported in the theory.} This is now stated explicitly in these corollaries. 
\vspace{4mm}

\item \emph{I suggest that the authors establish numerically the number of zero eigenvalues in figure 2 and figures 4 and 5 and update the paper and update the paper accordingly.}
This is now shown clearly in figures 3 and 6. (Figures have been renumbered in the revision). Figure 7 is a zoom around $\lambda = 1$ and does not show the origin. In addition, internal mode eigenvalues are shown on all figures, and markers have been updated for improved readability.
\vspace{4mm} 

\item \emph{Is it possible for this PDE to support other multi-pulse solutions that the components of these multi-pulse solutions are not almost equidistant?} Yes. This is illustrated for triple pulses in Figure 4.
\vspace{4mm}
\end{enumerate}

\closing{Sincerely,}

\end{letter}
\end{document}
